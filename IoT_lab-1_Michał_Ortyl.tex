\documentclass[10pt,a4paper]{article}
\usepackage[utf8]{inputenc}
\usepackage{amsmath}
\usepackage{amsfonts}
\usepackage{graphicx}
\usepackage{amssymb}
\usepackage{polski}
\usepackage{color}
\begin{document}
\setlength{\headheight}{10pt}

\begin{center}
\begin{tabular}{|c||c|} \hline


Politechnika Świetokrzystka w Kielach\\
Wydzial Elektorniki, Automatyki i Informatyki\\
\hline 
Laboratorium IOT \\
\hline 
Michał Ortyl \\ 
\hline 
Grupa : 15A \\
\hline
\end{tabular}

\section{Przykładowe rozmiary czcionek}
{\tiny Laboratorium 1 \\}
{\small Laboratorium 1 \\}
{\normalsize Laboratorium 1 \\}
{\large Laboratorium 1 \\}
{\huge Laboratorium 1 \\}
{\Huge Laboratorium 1 \\}
{\textit Laboratorium 1 \\}

\section{Rozmiary czcionek wraz z kolorami}
\begin{Large}
\textcolor{red}{LAboratorium IOT}\\
\end{Large}

\begin{huge}
\textcolor{green}{Laboratorium IOT}\\
\end{huge}
\end{center}
\begin{center}
\section{Przykładowa bilbiografia}
\end{center}
\cite{haken:atomy:kwanty}
\bibliographystyle{ieeetr}
\bibliography{references}

\newpage
\begin{center}
\section{Wnioski}
\end{center}

LaTeX – oprogramowanie do zautomatyzowanego składu tekstu, a także związany z nim język znaczników, służący do formatowania dokumentów tekstowych i tekstowo-graficznych (na przykład: broszur, artykułów, książek, plakatów, prezentacji, a nawet stron HTML)

Tworzenie tekstu w LaTeX-u opiera się na zasadzie WYSIWYM (What You See Is What You Mean - To, co widzisz, jest tym, o czym myślisz). Od zasady WYSIWYG odróżnia go to, że autor tekstu określa jedynie logiczną strukturę dokumentu (tzn. zaznacza, gdzie zaczyna się rozdział, co jest przypisem itp.), natomiast samym graficznym "ułożeniem" tekstu na stronie zajmuje się TeX, zwalniając tym samym użytkownika z tego zadania.

\end{document}
